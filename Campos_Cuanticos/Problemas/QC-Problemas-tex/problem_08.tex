Demostrar que las coordenadas de un vector posición $\bar{x}$ en la representación esférica $x_M$ pueden escribirse como:

$$
x_{M}=|\bar{x}| \sqrt{\frac{4 \pi}{3}} Y_{M}^{1}(\theta, \phi)
$$

\begin{solution}\ \\
$$
\begin{aligned}
&x_{t}=\mp\left(\frac{x \pm i y}{\sqrt{2}}\right) y \quad x_{0}=z \quad \text { Pasamos a esfericas }\\
&\left\{\begin{array}{l}
{x=r \cos \varphi \operatorname{sen} \theta} \\
{y=r \operatorname{sen} \varphi \operatorname{sen} \theta} \\
{z=r \cos \theta}
\end{array}\right.\\
&\left\{\begin{array}{l}
{x_{+}=\frac{-r}{\sqrt{2}} \operatorname{sen} \theta e^{i \varphi}} \\
{x_{-}=\frac{r}{\sqrt{2}} \operatorname{sen} \theta e^{-i \psi}} \\
{x_{0}=r \cos \theta}
\end{array}\right.
\end{aligned}
$$

$$
\left.\begin{array}{l}
{e^{4} \operatorname{sen} \theta=-\frac{8 \pi}{3} y_{1}^{1}(\theta, \varphi) \rightarrow x_{H}=r \sqrt{\frac{4 \pi}{3}} y_{1}^{\prime}(\theta, \varphi)} \\
{e^{-u} \operatorname{sen} \theta=\sqrt{\frac{8 \pi}{3}} y_{1}^{-1}\left(\theta_{1} y\right) \rightarrow x_{-1}=r \sqrt{\frac{4 \pi}{3}} y_{1}^{-1}(\theta, \varphi)} \\
{\cos \theta=\sqrt{\frac{4 \pi}{3}} y_{1}^{0}(\theta, \varphi) \longrightarrow x_{0}=r \sqrt{\frac{4 \pi}{3}} y_{1}^{\circ}(\theta, \varphi)}
\end{array}\right\}\left[x_{1}=r \sqrt{\frac{4 \pi}{3}} y_{1}^{m}(\theta, p)\right.
$$
\end{solution}