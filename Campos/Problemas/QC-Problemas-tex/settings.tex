\documentclass[12pt]{article}
\usepackage[margin=1in]{geometry} 
\usepackage[utf8]{inputenc} % allow utf-8 input
\usepackage[spanish]{babel}
\usepackage{amsmath}
%\usepackage{esvect}
\usepackage[a]{esvect}
\usepackage{verbatim} %% añadido de practica 2
\usepackage{amsgen,amsmath,amstext,amsbsy,amsopn,tikz,amssymb,tkz-linknodes} %% añadido de practica 2
\usetikzlibrary{arrows, calc, babel, shapes} %% añadido de practica 2
\usepackage{tcolorbox}
\usepackage{amssymb}
\usepackage{amsthm}
\usepackage{lastpage}
\usepackage{fancyhdr}
\usepackage{accents}
\usepackage{tkz-fct} %% añadido de practica 2
\usepackage{cancel}

\pagestyle{fancy}
\setlength{\headheight}{40pt}
\everymath{\displaystyle}

\newenvironment{solution}
  {\renewcommand\qedsymbol{$\blacksquare$}
  \begin{proof}[Solución]\noindent\rule{13cm}{0.4pt}}
  {\end{proof}\noindent\rule{\linewidth}{0.8pt}}
\renewcommand\qedsymbol{$\blacksquare$}

\newcommand{\ubar}[1]{\underaccent{\bar}{#1}}

%% añadido de practica 2:
\newcommand{\problem}[1]{
    {\begin{tikzpicture}[outline/.style={draw=red!75!gray,thick,fill=green!80!white}]
    \node [outline=red] at (0,1) {\bf Problema #1};
\end{tikzpicture}}
}