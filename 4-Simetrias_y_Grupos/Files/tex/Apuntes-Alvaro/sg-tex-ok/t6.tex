
\section{El grupo de Lorentz}

\subsection{Propiedades básicas}

El espacio de Minkowski es un espacio de $\mathds{R}^4$ con una métrica pseudoeuclídea de signatura (+,-,-,-). En una base ortonormal en la que los vectores contravariante $x^mu$ tienen coordenadas $x^\mu =(x^0,x^1,x^2,x^3)=(ct,x,y,z)=(x^0,\Vec{x})$ la métrica es diagonal.

$$\eta = \left ( \begin{array}{cccc}
 1& 0 & 0 & 0  \\
   0& -1 & 0 & 0  \\
     0& 0 & -1 & 0  \\
       0& 0 & 0 & -1
\end{array}\right)}$$

La norma al cuadrado del 4-vector se puede construir a través de la métrica como:

$$x^\mu \cdot x_\mu = x^\mu \eta _{\mu \nu} x_\nu= (x^0)^2 - (\Vec{x})^2$$

El grupo de Lorentz es el grupo de isometría de esta forma cuadrática (producto escalar), es decir, la norma en el espacio de Minkowski ha de ser invariante bajo transformaciones del grupo de Lorentz.

$$\Lambda \in O(1,3): \ \ \ x \ \longrightarrow x'=\Lambda x$$

$$x'\cdot x' =x\cdot x$$

La condición de invarianza de la métrica puede traducirse en $\Lambda ^\mu _\rho \eta _{\mu \nu} \Lambda ^\mu _\sigma =\eta _{\rho \sigma}$. O bien en forma matricial:

$$\Lambda ^t \eta \Lambda =\eta$$


Se puede demostrar que esas matrices $\Lambda$ forman un grupo bajo la multiplicación de matrices 4 $\times$ 4.

\smallskip
El grupo de Lorentz es un grupo de Lie lineal de dimensión 6 (6gdl). Es un grupo no conexo formado por 4 conjuntos disjuntos (componentes u hojas). Nos restringiremos a la hoja con determinante positivo y componente $\Lambda _0^0 \geq 0$ que es el llamado subgrupo ortocrono propio (o grupo de Lorentz restringido) $L__+^\uparrow \simeq$ SO(1,3).

\smallskip
\textbf{Ejercicio:} demuestra que el subgrupo de Lorentz ortocrono propio no solo es subgrupo si no que es subgrupo normal.

Otros subgrupos del grupo de Lorentz son el grupo propio $L_+^\uparrow U L_+^\downarrow$ y el subgrupo ortocrono $L_+^\uparrow U L_-^\uparrow$.

\subsection{Grupo de Lorentz ortocrono propio}

Todas las matrices $\Lambda \in L_+^\uparrow$ pueden escribirse como el producto de una rotación SO(3) por una transformación de Lorentz pura o boost. Veámoslas:

\begin{itemize}
\item \textbf{Rotaciones espaciales (R):} las rotaciones de ángulo $\phi$ respecto al eje $\Vec{n}$:

$$R_{\Vec{n}}(\phi) = \left( \begin{array}{c|ccc}
     1& 0 & 0 & 0  \\
     \hline
       0& - & - & -  \\
         0& - & R_\Vec{n}(\phi) & -  \\
           0& - & - & -
\end{array}\right)$$

En esta forma es evidente ver que SO(3) es subgrupo de $L_+^\uparrow$.

\newpage

\item \textbf{Transformación de Lorentz pura o boost  ($L$):} las trnasformaciones puras se hacen de velocidad v en la dirección de $\Vec{v}$. Por ejemplo, en la dirección x:

$$L_1=\left ( \begin{array}{cccc}
       \gamma & -\beta \gamma & 0 & 0  \\
        -\beta \gamma & \gamma & 0 & 0  \\
          0& 0 & 1 & 0  \\
            0& 0 & 0 & 1
\end{array}\right)$$

Utilizando el parámetro de boost $\uppsi$ se puede escribir $\gamma =cosh(\uppsi)$, $\beta = tanh(\uppsi)$ y queda:

$$L_1=\left ( \begin{array}{cccc}
       cosh(\uppsi) & -senh(\uppsi) & 0 & 0  \\
        -senh(\uppsi) & cosh(\uppsi) & 0 & 0  \\
          0& 0 & 1 & 0  \\
            0& 0 & 0 & 1
\end{array}\right)$$

En general, un boost en una dirección arbitraria podría escribirse de forma parecida aunque es bastante más sencillo obtenerlas a partir de las transformaciones infinitesimales así que no voy a copiar esto.

\smallskip
Se puede demostrar que cualquier transformación de Lorentz se puede obtener como una transformación de Lorentz en una dirección concreta, p.e $L_3$ con una rotación $R_{\Vec{n}}(\phi)$.

\smallskip
Además, en general, los boosts no forman subgrupo (solo las uniparamétricas entre ellas lo forman, por ejemplo $L_1$ consigo misma, $L_2$ consigo misma, etc).

\item \textbf{Parametrización:} tenemos que una transformación genérica se puede obtener de forma $\Lambda =LR$, de los 6 parámetros que forman $\Lambda$ se asocian 3 a las rotaciones y otras 3 a los boosts.

\smallskip
El espacio de parámetros correspondiente a las transformaciones L se puede tomar como un hiperboloide en el espacio euclídeo de 4 dimensiones (pues la norma de Minkowski es la ecuación de un hiperboloide). Solo tomamos la rama de arriba que es la que incluye el subgrupo ortocrono propio.

\smallskip
Es decir, los boosts nos desplazan por la rama del hiperboloide, entre dos puntos de esta. El subgrupo no es compacto (pues es un hiperboloide) y tampoco es simplemente conexo (será doblemente conexo) pues el subconjunto de transformaciones L si que es simplemente conexo (cualquier camino cerrado en el hiperboloide se puede encoger de forma continua a un punto) pero hereda de R, cuyas transformaciones son doblemente conexas su no conexión.

\item\textbf{Recubridor universal:} por definición, es recubridor universal es el grupo que siendo simplemente conexo sea homeomórfico a $L_+^\uparrow$. Asociamos a cada 4-vector $x=(x^0,x^1,x^2,x^3)$ la matriz hermítica 2 $\times$ 2:

$$X=\sigma _\mu x^\mu =\left ( \begin{array}{cc}
     x^0+x^3 & x^1-ix^2 \\
     x^1+ix^2 & x^0-x^3
\end{array}\right)$$

siendo $\sigma _\mu =(\mathds{1},\sigma _i)$ un vector con la matriz identidad y las matrices de pauli que cumple $\sigma ^\mu =\bar{\sigma}_\mu =(\sigma _0 , -\Vec{\sigma})$ y se verifica que Tr($\bar{\sigma}_\mu \sigma _\nu$)=$2\eta _{\mu \nu}$, que permite expresar $x^\mu =\frac{1}{2}Tr(\sigma ^\mu X)$.

Notemos que DetX=$x\cdot x$. Introduzcamos ahora una matriz compleja $A=\left ( \begin{array}{cc}
     \alpha & \beta \\
    \gamma & \delta
\end{array}\right)$ unimodular (det A=1) y transformemos X según:

$$X'=AXA^+$$

De modo que x'x'=xx preserve la norma. Esto demuestra que A describe una transformación lineal de $x^\mu$ que deja xx invariante (se corresponde con una transformación de Lorentz).

\newpage
La relación entre las matrices A y $\Lambda$ está dada por:

$$\Lambda _\nu ^\mu =\frac{1}{2}Tr(\bar{\sigma}^mu A \sigma _\nu A^+)$$

El conjunto de matrices A de componentes complejas y determinante 1 forman el grupo SL(2, $\mathds{C}$). Por cada matriz A hay una transformación de Lorentz $\Lambda$ y por cada transformación de Lorentz hay dos matrices de SL(2, $\mathds{C}$), A y -A. Luego es un homeomorfismo de grupos, en particular, tanto $\mathds{A}$ como $-\mathds{1}$ en SL(2, $\mathds{C}$) se corresponden con la identidad en $L_+^\uparrow$.

Esta correspondencia entre SL y $L_+^\uparrow$ es un homomorfismo ya que preserva la estructura de grupo.

$$S=S_1+S_2$$

$$S_1=-i\frac{\phi}{2}\Vec{n}\cdot \Vec{\sigma}, \ \ \ S_2=-\frac{\uppsi}{2}\Vec{u}\cdot \Vec{\sigma}$$

La exponenciación de $S_1$ nos da matrices unitarias (rotaciones) y la de $S_2$ nos da matrices hermíticas (boosts).

\smallskip
Hay un teorema del álgebra que garantiza que toda matriz de SL(2, $\mathds{C}$) admite descomposición polar (unitaria por hermítica $A=H\cdot U$ de manera única).

$$H=\sqrt{AA^+} \ \ \ U=H^{-1}A$$

El homomorfismo entre SL(2, $\mathds{C}$) y el grupo de Lorentz ortocrono propio nos garantiza que cualquier matriz del grupo $\lambda$ se puede descomponer en una rotación pura (unitaria) por un boost puro (hermítica).

\end{itemize}

\subsection{Álgebra de Lie de $L^\uparrow_+$}

Recordamos que las transformaciones infinitesimales eran los generadores del álgebra. Primeramente, el subgrupo de rotaciones espaciales puras generadas por tres elementos independientes es tomado como matrices ahora 4 $\times$ 4 de la forma:

$$\left (\begin{array}{cccc}
1 & 0 \\
0 & R_{i(3\times 3)}(\phi)
\end{array}\right )$$

Y tendrá tres generadores de las rotaciones puras, equivalentemente a lo que hemos visto en electromagnetismo:


$$J_1= i\left (\begin{array}{cccc}
0 & 0 &  0 &  0\\
0 & 0 &  0 & 0 \\
0 & 0 & 0 & -1\\
0 & 0 & 1 & 0
\end{array} \right )\ \ \ J_2= i\left (\begin{array}{cccc}
0 & 0 &  0 &  0\\
0 & 0 &  0 & 1 \\
0 & 0 & 0 & 0\\
0 & -1 & 0 & 0
\end{array} \right )\ \ \ J_3= i\left (\begin{array}{cccc}
0 & 0 &  0 &  0\\
0 & 0 &  -1 & 0 \\
0 & 1 & 0 & 0\\
0 & 0 & 0 & 0
\end{array} \right )$$


Una rotación infinitesimal de ángulo $\delta \phi$ en torno a un eje dado por $\Vec{n}$ es:

$$R=\mathds{1}-i\delta \phi \Vec{n}\cdot \Vec{J}$$

Análogamente para los boosts:

$$K_1= -i\left (\begin{array}{cccc}
0 & 1 &  0 &  0\\
1 & 0 &  0 & 0 \\
0 & 0 & 0 & 0\\
0 & 0 & 0 & 0
\end{array} \right ) \ \ \ K_2= -i\left (\begin{array}{cccc}
0 & 0 &  1 &  0\\
0 & 0 &  0 & 0 \\
1 & 0 & 0 & 0\\
0 & 0 & 0 & 0
\end{array} \right ) \ \ \ K_3= -i\left (\begin{array}{cccc}
0 & 0 &  0 &  1\\
0 & 0 &  0 & 0 \\
0 & 0 & 0 & 0\\
1 & 0 & 0 & 0
\end{array} \right )$$


Y por tanto el boost infinitesimal es:

$$L=\mathds{1}-i\delta \uppsi \Vec{u} \cdot \Vec{K}$$

\newpage

Las transformaciones finitas se obtienen, como ya deberíamos saber, de exponenciar el álgebra.

$$L=e^{i\uppsi \Vec{u} \cdot \Vec{K}}, \ \ \ R=e^{-i \phi \Vec{n} \cdot \Vec{J}} \longrightarrow \Lambda =e^{-i(\phi \Vec{n}\cdot \Vec{J} + \uppsi \Vec{u}\cdot \Vec{K})}$$

Lo que caracteriza estos generadores son sus conmutadores que son idénticos en todas sus representaciones. El álgebra está definida pues por los siguientes conmutadores:


$$[J_i,J_j]=i\epsilon _{ijk}J_k$$
$$[K_i,K_j]=-i\epsilon_{ijk}J_k$$

Es conveniente definir el tensor antisimétrico $M_{\mu \nu}$ de componentes:

$$(M_{12},M_{23},M_{31})=(J_3,J_1,J_2)$$
$$(M_{01},M_{02},M_{03})=(K_1,K_2,K_3)$$

Y se pueden agrupar las relaciones de conmutación:

$$[M_{\lambda \rho},M_{\mu \nu}]=-i[\eta_{\lambda \mu}M_{\rho \nu} + 2\eta _{\rho \nu} M_{\lambda \mu}-\eta _{\lambda \nu}M_{\rho \mu} - \eta_{\rho \mu}-\eta _{\rho \mu}M_{\lambda \nu} ]$$

Y en términos de este tensor se pueden escribir los generadores:

$$\Lambda = e^{\frac{-i}{2}\omega ^{\mu \nu} M_{\mu \nu}}$$

Los casimires también se pueden obtener en términos de este tensor:

$$\Vec{J}^2-\Vec{K}^2 = \frac{1}{2}M^{\mu \nu} M_{\mu \nu}, \ \ \ \frac{1}{2}\epsilon ^{\mu \nu \alpha \beta }M_{\mu \nu}M_{\alpha \beta}=-\Vec{J}\cdot \Vec{K}$$


\subsection{Representaciones irreducibles de $L^\uparrow _+$}

Queremos hallar las representaciones irreducibles de $L^\uparrow _+$ de dimensión finita. Dado que este grupo no es compacto, las representaciones no pueden ser unitarias ($\Lambda$ es combinación de unitaria y hermítica luego no es unitaria).

\smallskip
Para clasificar las representaciones irreducibles conviene introducir las siguientes combinaciones de generadores que son hermíticas.

$$\Vec{M}=\frac{\Vec{J+i\Vec{K}}}{2}, \ \ \ \Vec{N}=\frac{\Vec{J}-i\Vec{K}}{2}$$

Cumplen:

$$[M_i,M_j]=i\epsilon _{ijk}M_k, \ \ \ [N_i,N_j]=i\epsilon _{ijk}N_k, \ \ \ [M_i,N_j]=0$$

$\Vec{M}$ y $\Vec{N}$ son copias del álgebra de SU(2) solo que compleja. Dado que se etiquetar las representaciones irreducibles de SU(2) pues lo hemos hecho en el tema anterior, ahora es posible construir las de Lorentz a través de las de SU(2) (mediante estas combinaciones de generadores).

\smallskip
Denotando $M_\mathds{C}$ y $N_\mathds{C}$ a la envolvente lineal compleja de $\Vec{M}$ y $\Vec{N}$, tenemos el isomorfismo:

$$so(1,3)_\mathds{C}\simeq M_\mathds{C}\otimes N_\mathds{C} \equiv su(2)_\mathds{C}\otimes su(2)_\mathds{C} \equiv sl(2,\mathds{C}) \oplus sl(2,\mathds{C})$$

Concretamente mediante las relaciones de conmutación siguientes:

$$[\Vec{M}^2,M_i]=0, \ \ \ [\Vec{N}^2,N_i]=0$$

Podemos etiquetar las representaciones irreducibles de $L^\uparrow _+$ con los autovalores j(j+1) y j'(j'+1) de los operadores de casimir $M^2$ y $N^2$ respectivamente y construir exactamente la misma vaina que en SU(2).

Restringiéndonos solo a rotaciones puras, SO(3), las representaciones dejan de ser irreducibles y podrán descomponerse en términos de las representaciones irreducibles de SO(3).

$$D^{(j,j')}(R)=D^{(j)}(R)\otimesD^{(j')}(R)=D^{(j+j')}\oplus ... \oplus D^{|j-j'|}(R)$$

Dado que para SO(3) hay dos clases de representaciones irreducibles: tensoriales (con j+j' entero) y espinoriales (de j+j' semientero).

Las representaciones espinoriales se obtienen de su recubridor universal SL(2,$\mathds{C}$) y desde el punto de vista de $L^\uparrow _+$ son bivaluadas (por ser $L^\uparrow _+$ doblemente conexo, sus elementos admiten dos representaciones de SL(2,$\mathds{C}$)).

\smallskip
De acuerdo con esto, mientras que campos tensoriales se transforman bien bajo SO(1,3) los campos espinoriales se transforman bajo SL(2, $\mathds{C}$), análogamente a tensores y espinores de SO(3) y SU(2).

\subsubsection{Representaciones de espín más bajo}

Hay dos representaciones en la dimensión más baja j=0,j'=1/2 (recordar que la dimensión es (2j+1)(2j'+1)), o bien al contrario aunque no son equivalentes, veamos.

$$D^{(1/2,0)} \longrightarrow j=1/2, \ \ \ j'=0$$
$$D^{(0, 1/2)} \longrightarrow j=0, \ \ \ j'=1/2$$

Entendamos el origen de esta inequivalencia. Mientras que tenemos que $\Lambda (A)=\Lambda (-A)$ es la misma representación esto no ocurre así en general para A y $\bar{A}$ (A y su conjugado). Como en general A no es unitaria no existe una transformación de similaridad que relaciones A con $\bar{A}$ pero esto no pasa para las rotaciones para las que una representación y su conjugada sí que son equivalentes.

\smallskip
Por tanto, las matrices A y $\bar{A}$ constituyen de forma general representaciones irreducibles no equivalentes de $L^\uparrow _+$, que actúan en dos espacios vectoriales bidimensionales diferentes.

Tenemos entonces dos bases no equivalentes y en general dos clases de bi-espinores contravariantes que denotaremos $\xi$ y $\bar{\xi}$. Estos bi-espinores se transforman según:

$$\xi '=A\xi, \ \ \text{en componentes:} \ \ \ \xi =\left ( \begin{array}{c}
  \xi ^1  \\
 \xi ^2
\end{array}\right), \ \ \ \xi ^{\alpha '}=A ^{\alpha '}_\beta \xi ^\beta$$

$$\xi '=\bar{A}\xi, \ \ \text{en componentes:} \ \ \ \xi =\left ( \begin{array}{c}
  \xi ^\dot{1}  \\
 \xi ^\dot{2}
\end{array}\right), \ \ \ \xi ^{\dot{\alpha} '}=\bar{A} ^{\dot{\alpha} '}_{\dot{\beta}} \xi ^\dot{\beta}$$

En términos de componentes covariantes $\eta =(\eta _1 , \eta _2), \ \ \bar{\eta}=(\eta _{\dot{1}},\eta _{\dot{2}})$ se puede escribir:

$$\eta _{\alpha }' = \eta _{\beta} (A^{-1})^\beta _\alpha , \ \ \ \eta '_{\dot{\alpha}}=\eta _{\dot{ \beta}}(\bar{A}^{-1})^{\dot{\beta}}_{\dot{ \alpha}} \leftrightarrow \eta '=\eta A^{-1}, \ \ \ \bar{\eta}'=\bar{\eta}A^+$$

Y además sabemos que $\eta \xi$ y $ \bar{\eta}\bar{\xi}$ son invariantes.

\smallskip
\textbf{Ejercicio:} Una transformación de Lorentz general infinitesimal se puede escribir de la forma $\Lambda _\sigma ^\rho = \eta ^\rho _\sigma + \delta \omega ^\rho _\sigma$ con esta delta real infinitesimal. Hallar:

\begin{itemize}
\item Demostrar que $ \delta \omega ^\rho _\sigma$ es antisimétrica.

\item A partir de esta relación y de $\Lambda =e^{\frac{-1}{2}\omega ^{\mu \nu}M_{\mu \nu}}$ determinar los elementos de matriz de los generadores.

\item Verifica que se obtiene $M_{ij}=\epsilon _{ijk}J_k$ y $M_{0i}=K_i$.
\end{itemize}
