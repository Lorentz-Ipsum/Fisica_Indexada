\section{Elementos generales de teoría de grupos}

\begin{definicion}
$g_1$ se dice \textbf{conjugado} a $g_2$ si $\exists\ h$ tal que $g_1=hg_2h^{-1}$, con $g_1,g_2,h\in G$.
\begin{itemize}
\item Los elementos conjugados forman una \textbf{clase de conjugación}.
\item Si dos elementos son conjugados a un tercero, son conjugados entre sí. 
\item Cada elemento de un grupo forma parte de una única clase de conjugación.
\end{itemize}
\end{definicion}

\begin{definicion}
Un subgrupo $H$ de $G$ es \textbf{normal} o \textbf{invariante por conjugación}, $H\triangleleft G$, si 
\begin{equation}
ghg^{-1}\in H\qquad \forall g\in G\quad \forall h\in H
\end{equation}
\end{definicion}

\begin{teorema}
Un subgrupo es normal si es la unión de clases de conjugación.
\end{teorema}


 \begin{definicion}
 Un grupo se dice \textbf{simple} si no tiene subgrupos normales propios. Si no tiene subgrupos normales abelianos propios se dice \textbf{semi-simple}.
 \end{definicion}

\begin{definicion}
Dado un grupo $G$ con un subgrupo $H$, su \textbf{coset} por la izda. (resp. por la dcha.) asociado a $g\in  G$ es el \underline{conjunto} $gH=\left\{gh_i\right\}$ (resp. $Hg=\left\{h_i g\right\}$). 
\begin{itemize}
\item El coset es subgrupo ssi $g\in H$. 
\item Cada elemento de $G$ pertenece a algún coset. 
\item $G$ es la unión disjunta de cosets asociados a $H$.
\end{itemize}
\end{definicion}

\begin{teorema}[Lagrange] 
Si $G$ es un grupo finito y $H$ un subgrupo de $G$, el orden de $H$ es divisor del orden de $G$, i.e. $\abs{G}/\abs{H}$ es entero.
\end{teorema}

\begin{teorema}
Un subgrupo $H$ de un grupo $G$ es normal ssi sus cosets por la izda. y por la dcha. coinciden
\begin{equation}
H\triangleleft G \Longleftrightarrow gH=Hg \quad \forall g\in G
\end{equation}
\end{teorema}

\begin{definicion}
Definiendo el producto de dos cosets de $H\triangleleft G$ como
\begin{equation}
g_1 H \ast g_2 H=(g_1\cdot g_2) H
\end{equation}
donde $\cdot$ es la multiplicación ordinaria en $G$, el conjunto de cosets de $H$ forma un grupo con respecto a la multiplicación $\ast$, llamado \textbf{grupo cociente} y denotado por $G/H$. 
\end{definicion}

Si dos elementos $g,g'\in G$ pertenecen al mismo coset, $gH=g'H$, existe una relación de equivalencia entre ellos. El grupo cociente es el conjunto de estas clases de equivalencia. 
\begin{ejemplo}
Sea $G=S_n$ (grupo de las permutaciones) y $H=A_n$ (permutaciones pares). El grupo cociente clasifica a las permutaciones en pares e impares:
\begin{equation}
S_n/A_n=\left\{A_n,\tau_i A_n \right \}
\end{equation} \label{ejemplo_permutaciones_pares}
\end{ejemplo}


\begin{definicion}
Un \textbf{homomorfismo} es una aplicación que respeta la estructura de grupo:
\begin{subequations}
\begin{flalign}
&\phi: G\rightarrow G'\qquad (G,\cdot)\quad (G',\ast)\\
&\phi(g\cdot h)=\phi(g)\ast \phi(h)
\end{flalign}
\end{subequations}

En particular, se cumple
\begin{subequations}
\begin{gather}
\phi(e_G)=e_{G'}\\
\phi(g^{-1})=\left[\phi(g)\right]^{-1}
\end{gather}
\end{subequations}


Un \textbf{isomorfismo} es un homomorfismo biyectivo, $G\cong G'$
\end{definicion}

\begin{definicion}
El \textbf{núcleo} del homomorfismo es
\begin{equation}
\ker \phi= \phi^{-1}(e_{G'})=\left\{g\in G |\phi(g)=e_{G'}\right \}\subset G
\end{equation}
\end{definicion}

\begin{proposicion}
Un homomorfismo $\phi$ es \textbf{fiel} (inyectivo) ssi su núcleo es la identidad, $\ker\phi=\{e_G\}$.
\end{proposicion}

\begin{teorema}
Sea $\phi:G\rightarrow G'$ un homomorfismo entre grupos. Entonces:

\begin{enumerate}[label=\roman*)]
\item El núcleo es subgrupo normal de $G$, $\ker\phi \triangleleft G$
\item La imagen $\phi(G)$ es subgrupo de $G'$.
\item $G/\ker\phi \cong \phi(G)$ con el isomorfismo dado por
\begin{subequations}
\begin{flalign}
 \tilde{\phi}: G/\ker\phi &\rightarrow \phi(G)\\
 g\ker \phi &\mapsto \tilde{\phi} (g\ker\phi)=\phi (g)
\end{flalign}
\end{subequations}
\end{enumerate}


\end{teorema}

\begin{corolario}
Un subgrupo $H$ de un grupo $G$ es normal ssi existe un homomorfismo entre grupos $\phi:G\rightarrow G'$ con $\ker\phi=H$.
\end{corolario}

\begin{ejemplo}
El determinante es un homomorfismo de $\mathrm{GL}(n,\mathbb{C})$ en $\mathbb{C}^\ast=\mathbb{C}\backslash \left\{0\right\} $. Su núcleo está formado por el subgrupo normal de las matrices con $\det =1$

\begin{equation}
SL(n,\mathbb{C}) \triangleleft \mathrm{GL}(n,\mathbb{C})\qquad \mathrm{GL}(n,\mathbb{C})/SL(n,\mathbb{C})\cong \mathbb{C}^\ast %\backslash \left\{0\right\}
\end{equation}
\end{ejemplo}

\begin{definicion}
El \textbf{producto directo} de dos grupos $G_1$ y $G_2$ es el grupo
\begin{subequations}
\begin{flalign}
& G_1\times G_2= \left\{(g_1,g_2)|g_1\in G_1,\ g_2\in G_2\right\}\\
& (g_1,g_2)\star (g_1',g_2')= (g_1\cdot g_1', g_2\ast g_2')
\end{flalign}
\end{subequations}

\begin{itemize}
\item $\abs{G_1\times G_2}=\abs{G_1}\abs{G_2}$
\item $\left\{(g_1,e_{G_2})|g_1\in G_1\right\}\cong G_1$ y $\left\{(e_{G_1},g_2)|\ g_2\in G_2\right\} \cong G_2$ son subgrupos normales
\end{itemize}
\end{definicion}

\begin{teorema} \label{producto_directo_teorema}
Un grupo $G$ es \textbf{producto directo} de sus subgrupos $G_1$ y $G_2$ si cumple:
\begin{enumerate}[label=\roman*)]
\item $G_1$ y $G_2$ son subgrupos normales de $G$, o equivalentemente $g_1g_2=g_2g_1\ \forall g_i\in G_i$
\item Todo elemento $g\in G$ se puede expresar como $g=g_1g_2$ de forma única
\item $G_1\cap G_2=\{e\}$
\end{enumerate}
\end{teorema}

\begin{corolario}
Si $G=G_1\times G_2$, entonces $G/G_1\cong G_2$ y $G/G_2\cong G_1$.
\end{corolario}

\begin{definicion}
Un grupo $G$ es $\textbf{producto semi-directo}$, $G=G_1\rtimes G_2$ si %tiene dos subgrupos $G_1$ y $G_2$ tales que

\begin{enumerate}[label=\roman*)]
\item $G_1$ es subgrupo normal de $G$
\item Todo elemento $g\in G$ se puede expresar como $g=g_1g_2$ de forma única
\item $G_1\cap G_2=\{e\}$
\end{enumerate}

\end{definicion}