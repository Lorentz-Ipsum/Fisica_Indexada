% Paquetes de idioma y codificación de caracteres
\usepackage[english,spanish]{babel}
\usepackage[utf8]{inputenc}
\usepackage[T1]{fontenc}

% Paquetes de escritura matemática
\usepackage{amsmath,amsfonts,amssymb,bm,mathtools}

% Márgenes y tamaño del folio
\usepackage[left=2cm,right=2cm,top=2cm,bottom=2cm,a4paper]{geometry} 

% \usepackage{multirow} %Múltiples filas
% \usepackage{graphicx} %Gráficos
% \usepackage{soul} %Subrayar
% \usepackage{fancybox} %Cajitas guays
% \usepackage{multicol}
% \usepackage{cancel}
% \usepackage{pdfpages} %Inserción PDF
% \usepackage{tabu}%Ahorra tener que estar usando $ todo el rato en tablas
% \usepackage{realboxes}
% \usepackage{titlesec} %spacing before and after section titles
% \usepackage{lipsum} %texto de ejemplo
% % \usepackage{natbib}
% \usepackage{babelbib}
% \usepackage{urlbst}
\usepackage{siunitx}

\usepackage{color}
\usepackage{enumitem}% http://ctan.org/pkg/enumitem 
\usepackage{bbold} %identitity matrix symbol ("double" one)
	
	
% Definición de comandos: 
\newcommand{\abs}[1]{\left\lvert#1\right\rvert} %valor absoluto/módulo
\newcommand{\norm}[1]{\left\lVert#1\right\rVert} %norma
\newcommand{\refer}[1]{$^{[\ref{#1}]}$} %Referencias en superíndice
\newcommand{\uni}[1]{\ \mathrm{#1}} %Unidades con separación de un 'en' y  sin cursiva
\newcommand{\dif}{\mathrm{d} } %diferencial con d en roman
\newcommand{\der}[2]{\frac{\mathrm{d}#1}{\mathrm{d}#2}} %derivada de #1 resp a #2
\newcommand{\derd}[2]{\frac{\mathrm{d}^2#1}{\mathrm{d}#2^2}} %derivada segunda de #1 resp a #2
\newcommand{\dert}[2]{\frac{\mathrm{d}^3#1}{\mathrm{d}#2^3}} %derivada tercera de #1 resp a #2
\newcommand{\parc}[2]{\frac{\partial#1}{\partial#2}} %derivada parcial de #1 resp a #2
\newcommand{\parcd}[2]{\frac{\partial^2#1}{\partial#2^2}} %derivada de #1 resp a #2
\newcommand{\curl}{\nabla\times} %curl operator
\newcommand{\rot}{\nabla\times}  %operador rotacional
\newcommand{\diver}{\nabla\cdot} %operador divergencia
\newcommand{\laplac}{\nabla^2}
\newcommand{\mean}[1]{\langle #1\rangle } %Media
\newcommand{\pe}[2]{\langle #1|#2\rangle } %Producto escalar
\newcommand{\bra}[1]{\left\langle#1\right|}  %bra
\newcommand{\ket}[1]{\left|#1\right\rangle} %ket
\newcommand{\braket}[2]{\left\langle#1\right|\left.#2\right\rangle} 
% \newcommand{\braopket}[3]{\left\langle#1\right.\left|#2\right|\left.#3\right\rangle}
\newcommand{\braopket}[3]{\bra{#1}#2\ket{#3}}
\newcommand{\parder}[2]{\frac{\partial#1}{\partial#2}} %derivada parcial de #1 resp a #2
\newcommand{\parderd}[2]{\frac{\partial^2#1}{\partial#2^2}} %derivada parcial segunda de #1 resp a #2
\newcommand{\trace}[1]{\operatorname{Tr}\left(#1\right)} %traza
% \newcommand{\suchthat}{\mathrel{\mathop\supset}\kern-4.0pt$-$\kern-1.0pt$-~$}

\usepackage{mathrsfs}
% \newcommand{\lie}{\mathcal{L}} %álgebra de Lie
\newcommand{\lie}{\mathscr{L}} %álgebra de Lie

% Definiciones y teoremas
% https://www.overleaf.com/learn/latex/Theorems_and_proofs
% https://tex.stackexchange.com/questions/283578/how-to-produce-theorems-in-which-the-color-of-heading-and-body-of-the-theorem-my
\usepackage{amsthm,thmtools,xcolor}

\theoremstyle{definition}
\newtheorem{definicion}{Definición}[section]
% \theoremstyle{definition}
\newtheorem{teorema}{Teorema}[section]
\newtheorem{proposicion}[teorema]{Proposición}
\newtheorem{lema}[teorema]{Lema}
\newtheorem{corolario}{Corolario}[teorema]
\newtheorem{relacion}[teorema]{Relación}%[section]

% \theoremstyle{remark}
\newtheorem{remark}{Remark}[section]
\newtheorem{observacion}[remark]{Observación}%[section]
\newtheorem{nota}[remark]{Nota}%[section]
% \newtheorem{ejemplo}{Ejemplo}[section]


% examples in blue
\declaretheoremstyle[
  headfont=\color{blue}\normalfont\bfseries,
  bodyfont=\color{blue}\normalfont,
]{colored}
\declaretheorem[
  style=colored,
  numberwithin=section,
  name=Ejemplo,
]{ejemplo}





% Primer tema omitido
\newcommand{\fakesection}[1]{%
  \par\refstepcounter{section}% Increase section counter
  \sectionmark{#1}% Add section mark (header)
  \addcontentsline{toc}{section}{\protect\numberline{\thesection}#1}% Add section to ToC
  % Add more content here, if needed.
}


% Página en blanco 
\usepackage{afterpage}
\newcommand\blankpage{%
	\null
	\thispagestyle{empty}%
	\addtocounter{page}{-1}%
	\newpage}
	
% Numeración de las ecuaciones por secciones
\numberwithin{equation}{section}

